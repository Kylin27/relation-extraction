\documentclass[mathserif]{beamer}
\usetheme{Berkeley}
\usecolortheme{albatross}

\title{Constructing a Knowledge Base on Aging}
\subtitle{An Automated Approach}
\author{Mark Farrell}
\institute[UWaterloo and UC Berkeley]
{
  \inst{}%
  Computer Science Club \\
  University of Waterloo \and

  \inst{}%
  Center for Research and Education on Aging \\
  Lawrence Berkeley National Laboratory \\
  University of California, Berkeley

}

\date{September 4th, 2014}
\subject{Natural Language Processing}

\AtBeginSection[]
{
  \begin{frame}
    \frametitle{Outline}
    \tableofcontents[currentsection]
  \end{frame}
}

\begin{document}

  \frame{\titlepage}

  \section{Automatically Constructing Knowledge Bases}
  \begin{frame}

    \frametitle{Overview}

    \begin{itemize}[<+->]

      \item CREA is constructing a knowledge base to study and
      understand the human aging process.
      \item New discoveries are published quickly and in large
      volume.
      \item It is infeasible to construct the knowledge base by
      hand.
      \item Working on software to construct the knowledge base
      automatically.

    \end{itemize}

  \end{frame}

  \begin{frame}

    \frametitle{Introduction}
    \framesubtitle{How to Automatically Construct the Knowledge Base}

    \begin{itemize}[<+->]

      \item Routinely search for keywords related to aging, dowloading
      text articles from sources like PubMed and WebMD.
      \item Build a spam filter to get rid of non-scientific sentences.
      \item Extract scientific facts from the sentences and save them
      in a structured format.
      \item Provide a graphical interface that allows users to search
      and otherwise explore the knowledge base.

    \end{itemize}

  \end{frame}

  \begin{frame}

    \frametitle{Summary of Progress}

    \begin{itemize}[<+->]

      \item Devised and implemented the method for finding simple facts
      in sentences, extracting them in a structured format.

      % Projects like Carnegie Mellon's "Read The Web" appear to be able
      % to extract simple facts from sentences, but their software is
      % neither open-source nor downloadable.

      \item Began work on a web viewer for the knowledge base.

    \end{itemize}

  \end{frame}

  \section{Extracting Facts in a Structured Format}
  \begin{frame}

    \frametitle{Tokenization}

    \begin{itemize}[<+->]

      \item Input a text document and read it, one sentence at a time.

    \end{itemize}

  \end{frame}

  \begin{frame}

    \frametitle{Parsing}

    \begin{itemize}[<+->]

      \item For each sentence, generate a constituent tree that describes its phrase structure.

      \item The University of Pennsylvania Treebank Project:
      \begin{itemize}[<+->]
        \item Defines notation for constituent trees.
        \item Parses sentences from the Wall Street Journal by hand.
      \end{itemize}

      \item The Berkeley Parser is software that guesses how to parse a sentence from that notation
      and examples specified by the Penn Treebank.

    \end{itemize}

  \end{frame}

  \begin{frame}

    \frametitle{Compilation}

    \begin{itemize}[<+->]

      \item Construct facts from each constituent tree. Start at the bottom:
      \begin{itemize}[<+->]

        \item First look for nouns.
        \item Then look for predicates, an action being performed on
        one or more nouns.
        \item Output the facts, describing the actions found in each
        sentence and the nouns that they are performed on.

      \end{itemize}

    \end{itemize}

  \end{frame}

  \section{Results}

  \begin{frame}

    \frametitle{Parallelization}

    \begin{itemize}[<+->]

      \item It is possible to extract facts from many sentences at the same time.

    \end{itemize}

  \end{frame}

  \section{Discussion}

  \begin{frame}

    \frametitle{Discussion}

    \begin{itemize}[<+->]

       \item Define more patterns for extracting facts:
       \begin{itemize}[<+->]

          \item The software succeeds around 50\% of the time.

       \end{itemize}

       \item The accuracy of the parser could be optimized:
       \begin{itemize}[<+->]

          \item Should be trained to identify more nouns from the biomedical
          domain.

       \end{itemize}

       \item Support negated clauses and conditional logic.

       \item Facts can contradict each other: store the probability that
       is true as the weight of its edge on the knowledge base's graph.

       \item Filter spam sentences from documents.

       \item Scale and launch the software service, automatically
       constructing CREA's knowledge base on aging.

    \end{itemize}

  \end{frame}

\end{document}
